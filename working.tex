\documentclass{article}
\usepackage[utf8]{inputenc}
\usepackage{kotex}



\begin{document}

\title{K학술확산연구소 홈페이지 기획서}
\author{박석훈}
\date{March 2022}
\maketitle

\tableofcontents

\begin{abstract}
    한국의 경제발전 경험에 대한 학술적 연구 성과를 공유하고 확산하는 것을 목표로 하되, 한국경제사 연구의 최신 성과를 다각적으로 조명한다. 해방 이후 75년간 한국 경제는 한강의 기적(Miracle on the Han River)이라고 불릴 정도로 눈부신 고도성장을 하였으며, 아시아의 네 마리 호랑이(Four Asian Tigers) 중의 하나인 신흥공업국(NICs)으로서 주목받은 바 있다. 한국이 OECD 가입국이자 DAC 멤버로서 선진국 반열에 이르게 된 역사적 경험에 대해 그간 축적된 학술 연구 성과를 국내외에 알림으로써, 한국이 경제뿐 아니라 학술 측면에서도 세계를 선도하는 위상을 가질 수 있도록 한다. \\
\end{abstract}



\section{연구 배경}
ㄹㄹ

\section{주관 연구기관 개요}
\subsection{주관연구기관의 한국학 연구 · 교육 현황}
• 한국학 연구총서·학술지 현황
- 연구총서: 한국학연구총서 44권, 한국학모노그래프 70권, 한국학공동연구총서 13권, 한국문화연구총서 36권, 규장각학술총서 13권 등
- 학술지: Seoul Journal of Korean Studies, Seoul Journal of Economics, Seoul Journal of Business, Journal of Korean Law, ��한국문화��, ��한국정치연구�� 등
• 한국학 관련 연구소 운영 현황
- 한국경제 전문 연구기관인 경제연구소의 설치·운영: 1961년 설립, 1962년부터 국문 학술지 간행, 1988년부터 영문 학술지 간행 ⇒ 미래지향적 연구 수행을 위해 한국경제혁신센터를 설치(건물 신축 확정, 2021년 설계, 2022년 착공)
- 한국학 전문기관인 규장각한국학연구원의 설치·운영: 교육·연구 및 아카이브
- 한국학 관련 기타 연구소: 인문학연구원, 한국어문학연구소, 역사연구소, 통일평화연구원, 사회과학연구원, 법학연구소, 한국행정연구소, 국토문제연구소 등
• 한국학 관련 교육 현황
- 국제대학원의 한국학 교육: 국제지역학(한국지역) 외에 석·박사과정의 Korean Studies 프로그램을 운영(문학, 역사, 정치, 외교, 사회, 경제, 문화, 예술 등)
- 언어교육원의 한국어 교육: 한국어교육센터, 스마트 한국어과정 등
- 평생교육원의 프로그램: 한국어교원 양성과정, 한국어교육 전문지도자 양성과정
- 규장각의 정례적 교육 행사: 금요시민강좌, 한문강좌, 한문워크숍, 여름학교 등
• 한국학 관련 해외교류 현황
- 국제한국학센터를 통해 외국인 교육·연구를 포함한 해외 교류 활동: 해외 한국학 저자특강, 해외 한국학 사서 워크숍, 한국학 국제학술대회 등
- 국제협력본부의 국제교육프로그램: SNU in the World Program, Study Abroad Program(SAP), 국제하계강좌(International Summer Program, ISP) 등 ⇒ 2021학년도 ISP에는 기존에 운영되던 교과목 10개와 더불어 한국학, 국제·동아시아 연구, 경제·비즈니스, 공학·생명과학 등의 분야에서 15개 이상의 정규 전공 과정이 추가됨


\section{연구 내용}


\section{홈페이지 구축}
\subsection{구조도}





\subsection{페이지 상세설명}

\end{document}
